% Options for packages loaded elsewhere
\PassOptionsToPackage{unicode}{hyperref}
\PassOptionsToPackage{hyphens}{url}
\PassOptionsToPackage{dvipsnames,svgnames,x11names}{xcolor}
%
\documentclass[
  letterpaper,
  DIV=11,
  numbers=noendperiod]{scrartcl}

\usepackage{amsmath,amssymb}
\usepackage{iftex}
\ifPDFTeX
  \usepackage[T1]{fontenc}
  \usepackage[utf8]{inputenc}
  \usepackage{textcomp} % provide euro and other symbols
\else % if luatex or xetex
  \usepackage{unicode-math}
  \defaultfontfeatures{Scale=MatchLowercase}
  \defaultfontfeatures[\rmfamily]{Ligatures=TeX,Scale=1}
\fi
\usepackage{lmodern}
\ifPDFTeX\else  
    % xetex/luatex font selection
\fi
% Use upquote if available, for straight quotes in verbatim environments
\IfFileExists{upquote.sty}{\usepackage{upquote}}{}
\IfFileExists{microtype.sty}{% use microtype if available
  \usepackage[]{microtype}
  \UseMicrotypeSet[protrusion]{basicmath} % disable protrusion for tt fonts
}{}
\makeatletter
\@ifundefined{KOMAClassName}{% if non-KOMA class
  \IfFileExists{parskip.sty}{%
    \usepackage{parskip}
  }{% else
    \setlength{\parindent}{0pt}
    \setlength{\parskip}{6pt plus 2pt minus 1pt}}
}{% if KOMA class
  \KOMAoptions{parskip=half}}
\makeatother
\usepackage{xcolor}
\setlength{\emergencystretch}{3em} % prevent overfull lines
\setcounter{secnumdepth}{-\maxdimen} % remove section numbering
% Make \paragraph and \subparagraph free-standing
\makeatletter
\ifx\paragraph\undefined\else
  \let\oldparagraph\paragraph
  \renewcommand{\paragraph}{
    \@ifstar
      \xxxParagraphStar
      \xxxParagraphNoStar
  }
  \newcommand{\xxxParagraphStar}[1]{\oldparagraph*{#1}\mbox{}}
  \newcommand{\xxxParagraphNoStar}[1]{\oldparagraph{#1}\mbox{}}
\fi
\ifx\subparagraph\undefined\else
  \let\oldsubparagraph\subparagraph
  \renewcommand{\subparagraph}{
    \@ifstar
      \xxxSubParagraphStar
      \xxxSubParagraphNoStar
  }
  \newcommand{\xxxSubParagraphStar}[1]{\oldsubparagraph*{#1}\mbox{}}
  \newcommand{\xxxSubParagraphNoStar}[1]{\oldsubparagraph{#1}\mbox{}}
\fi
\makeatother

\usepackage{color}
\usepackage{fancyvrb}
\newcommand{\VerbBar}{|}
\newcommand{\VERB}{\Verb[commandchars=\\\{\}]}
\DefineVerbatimEnvironment{Highlighting}{Verbatim}{commandchars=\\\{\}}
% Add ',fontsize=\small' for more characters per line
\usepackage{framed}
\definecolor{shadecolor}{RGB}{241,243,245}
\newenvironment{Shaded}{\begin{snugshade}}{\end{snugshade}}
\newcommand{\AlertTok}[1]{\textcolor[rgb]{0.68,0.00,0.00}{#1}}
\newcommand{\AnnotationTok}[1]{\textcolor[rgb]{0.37,0.37,0.37}{#1}}
\newcommand{\AttributeTok}[1]{\textcolor[rgb]{0.40,0.45,0.13}{#1}}
\newcommand{\BaseNTok}[1]{\textcolor[rgb]{0.68,0.00,0.00}{#1}}
\newcommand{\BuiltInTok}[1]{\textcolor[rgb]{0.00,0.23,0.31}{#1}}
\newcommand{\CharTok}[1]{\textcolor[rgb]{0.13,0.47,0.30}{#1}}
\newcommand{\CommentTok}[1]{\textcolor[rgb]{0.37,0.37,0.37}{#1}}
\newcommand{\CommentVarTok}[1]{\textcolor[rgb]{0.37,0.37,0.37}{\textit{#1}}}
\newcommand{\ConstantTok}[1]{\textcolor[rgb]{0.56,0.35,0.01}{#1}}
\newcommand{\ControlFlowTok}[1]{\textcolor[rgb]{0.00,0.23,0.31}{\textbf{#1}}}
\newcommand{\DataTypeTok}[1]{\textcolor[rgb]{0.68,0.00,0.00}{#1}}
\newcommand{\DecValTok}[1]{\textcolor[rgb]{0.68,0.00,0.00}{#1}}
\newcommand{\DocumentationTok}[1]{\textcolor[rgb]{0.37,0.37,0.37}{\textit{#1}}}
\newcommand{\ErrorTok}[1]{\textcolor[rgb]{0.68,0.00,0.00}{#1}}
\newcommand{\ExtensionTok}[1]{\textcolor[rgb]{0.00,0.23,0.31}{#1}}
\newcommand{\FloatTok}[1]{\textcolor[rgb]{0.68,0.00,0.00}{#1}}
\newcommand{\FunctionTok}[1]{\textcolor[rgb]{0.28,0.35,0.67}{#1}}
\newcommand{\ImportTok}[1]{\textcolor[rgb]{0.00,0.46,0.62}{#1}}
\newcommand{\InformationTok}[1]{\textcolor[rgb]{0.37,0.37,0.37}{#1}}
\newcommand{\KeywordTok}[1]{\textcolor[rgb]{0.00,0.23,0.31}{\textbf{#1}}}
\newcommand{\NormalTok}[1]{\textcolor[rgb]{0.00,0.23,0.31}{#1}}
\newcommand{\OperatorTok}[1]{\textcolor[rgb]{0.37,0.37,0.37}{#1}}
\newcommand{\OtherTok}[1]{\textcolor[rgb]{0.00,0.23,0.31}{#1}}
\newcommand{\PreprocessorTok}[1]{\textcolor[rgb]{0.68,0.00,0.00}{#1}}
\newcommand{\RegionMarkerTok}[1]{\textcolor[rgb]{0.00,0.23,0.31}{#1}}
\newcommand{\SpecialCharTok}[1]{\textcolor[rgb]{0.37,0.37,0.37}{#1}}
\newcommand{\SpecialStringTok}[1]{\textcolor[rgb]{0.13,0.47,0.30}{#1}}
\newcommand{\StringTok}[1]{\textcolor[rgb]{0.13,0.47,0.30}{#1}}
\newcommand{\VariableTok}[1]{\textcolor[rgb]{0.07,0.07,0.07}{#1}}
\newcommand{\VerbatimStringTok}[1]{\textcolor[rgb]{0.13,0.47,0.30}{#1}}
\newcommand{\WarningTok}[1]{\textcolor[rgb]{0.37,0.37,0.37}{\textit{#1}}}

\providecommand{\tightlist}{%
  \setlength{\itemsep}{0pt}\setlength{\parskip}{0pt}}\usepackage{longtable,booktabs,array}
\usepackage{calc} % for calculating minipage widths
% Correct order of tables after \paragraph or \subparagraph
\usepackage{etoolbox}
\makeatletter
\patchcmd\longtable{\par}{\if@noskipsec\mbox{}\fi\par}{}{}
\makeatother
% Allow footnotes in longtable head/foot
\IfFileExists{footnotehyper.sty}{\usepackage{footnotehyper}}{\usepackage{footnote}}
\makesavenoteenv{longtable}
\usepackage{graphicx}
\makeatletter
\def\maxwidth{\ifdim\Gin@nat@width>\linewidth\linewidth\else\Gin@nat@width\fi}
\def\maxheight{\ifdim\Gin@nat@height>\textheight\textheight\else\Gin@nat@height\fi}
\makeatother
% Scale images if necessary, so that they will not overflow the page
% margins by default, and it is still possible to overwrite the defaults
% using explicit options in \includegraphics[width, height, ...]{}
\setkeys{Gin}{width=\maxwidth,height=\maxheight,keepaspectratio}
% Set default figure placement to htbp
\makeatletter
\def\fps@figure{htbp}
\makeatother

\KOMAoption{captions}{tableheading}
\makeatletter
\@ifpackageloaded{caption}{}{\usepackage{caption}}
\AtBeginDocument{%
\ifdefined\contentsname
  \renewcommand*\contentsname{Table of contents}
\else
  \newcommand\contentsname{Table of contents}
\fi
\ifdefined\listfigurename
  \renewcommand*\listfigurename{List of Figures}
\else
  \newcommand\listfigurename{List of Figures}
\fi
\ifdefined\listtablename
  \renewcommand*\listtablename{List of Tables}
\else
  \newcommand\listtablename{List of Tables}
\fi
\ifdefined\figurename
  \renewcommand*\figurename{Figure}
\else
  \newcommand\figurename{Figure}
\fi
\ifdefined\tablename
  \renewcommand*\tablename{Table}
\else
  \newcommand\tablename{Table}
\fi
}
\@ifpackageloaded{float}{}{\usepackage{float}}
\floatstyle{ruled}
\@ifundefined{c@chapter}{\newfloat{codelisting}{h}{lop}}{\newfloat{codelisting}{h}{lop}[chapter]}
\floatname{codelisting}{Listing}
\newcommand*\listoflistings{\listof{codelisting}{List of Listings}}
\makeatother
\makeatletter
\makeatother
\makeatletter
\@ifpackageloaded{caption}{}{\usepackage{caption}}
\@ifpackageloaded{subcaption}{}{\usepackage{subcaption}}
\makeatother

\ifLuaTeX
  \usepackage{selnolig}  % disable illegal ligatures
\fi
\usepackage{bookmark}

\IfFileExists{xurl.sty}{\usepackage{xurl}}{} % add URL line breaks if available
\urlstyle{same} % disable monospaced font for URLs
\hypersetup{
  pdftitle={class 6: R functions},
  pdfauthor={Mia Fava},
  colorlinks=true,
  linkcolor={blue},
  filecolor={Maroon},
  citecolor={Blue},
  urlcolor={Blue},
  pdfcreator={LaTeX via pandoc}}


\title{class 6: R functions}
\author{Mia Fava}
\date{}

\begin{document}
\maketitle


Today we are going to explore R functions and beging think about writing
our own functions

Let's start simple and write our first function to ass some numbers

Every function in R has at least 3 things

-a \textbf{name}, we pick this -one or more input \textbf{arguments}
-the \textbf{body}, where the work actually happens

\begin{Shaded}
\begin{Highlighting}[]
\NormalTok{add }\OtherTok{\textless{}{-}} \ControlFlowTok{function}\NormalTok{(x,}\AttributeTok{y=}\DecValTok{1}\NormalTok{, }\AttributeTok{z=}\DecValTok{0}\NormalTok{)\{}
\NormalTok{  x }\SpecialCharTok{+}\NormalTok{ y }
\NormalTok{\}}
\end{Highlighting}
\end{Shaded}

Now lets try it out

\begin{Shaded}
\begin{Highlighting}[]
\FunctionTok{add}\NormalTok{(}\FunctionTok{c}\NormalTok{(}\DecValTok{10}\NormalTok{,}\DecValTok{1}\NormalTok{,}\DecValTok{1}\NormalTok{,}\DecValTok{10}\NormalTok{),}\DecValTok{1}\NormalTok{)}
\end{Highlighting}
\end{Shaded}

\begin{verbatim}
[1] 11  2  2 11
\end{verbatim}

\begin{Shaded}
\begin{Highlighting}[]
\FunctionTok{add}\NormalTok{ (}\DecValTok{10}\NormalTok{)}
\end{Highlighting}
\end{Shaded}

\begin{verbatim}
[1] 11
\end{verbatim}

\begin{Shaded}
\begin{Highlighting}[]
\FunctionTok{add}\NormalTok{ (}\DecValTok{10}\NormalTok{,}\DecValTok{20}\NormalTok{)}
\end{Highlighting}
\end{Shaded}

\begin{verbatim}
[1] 30
\end{verbatim}

\begin{Shaded}
\begin{Highlighting}[]
\FunctionTok{add}\NormalTok{ (}\DecValTok{10}\NormalTok{,}\DecValTok{10}\NormalTok{,}\DecValTok{20}\NormalTok{)}
\end{Highlighting}
\end{Shaded}

\begin{verbatim}
[1] 20
\end{verbatim}

\begin{Shaded}
\begin{Highlighting}[]
\FunctionTok{mean}\NormalTok{(}\FunctionTok{c}\NormalTok{(}\DecValTok{10}\NormalTok{,}\DecValTok{10}\NormalTok{,}\ConstantTok{NA}\NormalTok{),}\AttributeTok{na.rm=}\ConstantTok{TRUE}\NormalTok{)}
\end{Highlighting}
\end{Shaded}

\begin{verbatim}
[1] 10
\end{verbatim}

\subsection{Lab sheet work}\label{lab-sheet-work}

\begin{quote}
Q1 Write a function grade() to determine an overall grade from a vector
of student homework assignment scores dropping the lowest single score.
If a student misses a homework (i.e.~has an NA value) this can be used
as a score to be potentially dropped. Your final function should be
adquately explained with code comments and be able to work on an example
class gradebook such as this one in CSV format:
``https://tinyurl.com/gradeinput'' {[}3pts{]}
\end{quote}

\begin{Shaded}
\begin{Highlighting}[]
\CommentTok{\# Example input vectors to start with}
\NormalTok{student1 }\OtherTok{\textless{}{-}} \FunctionTok{c}\NormalTok{(}\DecValTok{100}\NormalTok{, }\DecValTok{100}\NormalTok{, }\DecValTok{100}\NormalTok{, }\DecValTok{100}\NormalTok{, }\DecValTok{100}\NormalTok{, }\DecValTok{100}\NormalTok{, }\DecValTok{100}\NormalTok{, }\DecValTok{90}\NormalTok{)}
\NormalTok{student2 }\OtherTok{\textless{}{-}} \FunctionTok{c}\NormalTok{(}\DecValTok{100}\NormalTok{, }\ConstantTok{NA}\NormalTok{, }\DecValTok{90}\NormalTok{, }\DecValTok{90}\NormalTok{, }\DecValTok{90}\NormalTok{, }\DecValTok{90}\NormalTok{, }\DecValTok{97}\NormalTok{, }\DecValTok{80}\NormalTok{)}
\NormalTok{student3 }\OtherTok{\textless{}{-}} \FunctionTok{c}\NormalTok{(}\DecValTok{90}\NormalTok{, }\ConstantTok{NA}\NormalTok{, }\ConstantTok{NA}\NormalTok{, }\ConstantTok{NA}\NormalTok{, }\ConstantTok{NA}\NormalTok{, }\ConstantTok{NA}\NormalTok{, }\ConstantTok{NA}\NormalTok{, }\ConstantTok{NA}\NormalTok{)}
\end{Highlighting}
\end{Shaded}

Begin by calculating the average for student1

\begin{Shaded}
\begin{Highlighting}[]
\NormalTok{student1}
\end{Highlighting}
\end{Shaded}

\begin{verbatim}
[1] 100 100 100 100 100 100 100  90
\end{verbatim}

\begin{Shaded}
\begin{Highlighting}[]
\FunctionTok{mean}\NormalTok{(student1)}
\end{Highlighting}
\end{Shaded}

\begin{verbatim}
[1] 98.75
\end{verbatim}

try on student2

\begin{Shaded}
\begin{Highlighting}[]
\NormalTok{student2}
\end{Highlighting}
\end{Shaded}

\begin{verbatim}
[1] 100  NA  90  90  90  90  97  80
\end{verbatim}

\begin{Shaded}
\begin{Highlighting}[]
\FunctionTok{mean}\NormalTok{(student2,}\AttributeTok{na.rm=}\ConstantTok{TRUE}\NormalTok{)}
\end{Highlighting}
\end{Shaded}

\begin{verbatim}
[1] 91
\end{verbatim}

and student3

\begin{Shaded}
\begin{Highlighting}[]
\NormalTok{student3}
\end{Highlighting}
\end{Shaded}

\begin{verbatim}
[1] 90 NA NA NA NA NA NA NA
\end{verbatim}

\begin{Shaded}
\begin{Highlighting}[]
\FunctionTok{mean}\NormalTok{(student3,}\AttributeTok{na.rm=}\ConstantTok{TRUE}\NormalTok{)}
\end{Highlighting}
\end{Shaded}

\begin{verbatim}
[1] 90
\end{verbatim}

Hmm\ldots this sucks! I need to find something else and back to this
issue of missing values (NAs).

We also want to drop the lowest score from a given student set of
scores.

\begin{Shaded}
\begin{Highlighting}[]
\NormalTok{student1}
\end{Highlighting}
\end{Shaded}

\begin{verbatim}
[1] 100 100 100 100 100 100 100  90
\end{verbatim}

\begin{Shaded}
\begin{Highlighting}[]
\NormalTok{student1[}\SpecialCharTok{{-}}\DecValTok{8}\NormalTok{]}
\end{Highlighting}
\end{Shaded}

\begin{verbatim}
[1] 100 100 100 100 100 100 100
\end{verbatim}

We can try the \texttt{min()} function to find lowest score

\begin{Shaded}
\begin{Highlighting}[]
\FunctionTok{min}\NormalTok{(student1)}
\end{Highlighting}
\end{Shaded}

\begin{verbatim}
[1] 90
\end{verbatim}

We can also try \texttt{which.min()} to find the location of the lowest
score, not the value itself

\begin{Shaded}
\begin{Highlighting}[]
\FunctionTok{which.min}\NormalTok{(student1)}
\end{Highlighting}
\end{Shaded}

\begin{verbatim}
[1] 8
\end{verbatim}

Lets put these two things together

\begin{Shaded}
\begin{Highlighting}[]
\NormalTok{student1[}\FunctionTok{which.min}\NormalTok{(student1)]}
\end{Highlighting}
\end{Shaded}

\begin{verbatim}
[1] 90
\end{verbatim}

\begin{Shaded}
\begin{Highlighting}[]
\FunctionTok{mean}\NormalTok{(student1[}\FunctionTok{which.min}\NormalTok{(student1)])}
\end{Highlighting}
\end{Shaded}

\begin{verbatim}
[1] 90
\end{verbatim}

We need to deal with NA (missing values) somehow?

One idea is we make all the NA values zero

\begin{Shaded}
\begin{Highlighting}[]
\NormalTok{x }\OtherTok{\textless{}{-}}\NormalTok{ student2}
\NormalTok{x}
\end{Highlighting}
\end{Shaded}

\begin{verbatim}
[1] 100  NA  90  90  90  90  97  80
\end{verbatim}

\begin{Shaded}
\begin{Highlighting}[]
\NormalTok{x[}\DecValTok{2}\NormalTok{]}\OtherTok{\textless{}{-}} \DecValTok{0}
\NormalTok{x}
\end{Highlighting}
\end{Shaded}

\begin{verbatim}
[1] 100   0  90  90  90  90  97  80
\end{verbatim}

\begin{Shaded}
\begin{Highlighting}[]
\NormalTok{x }\OtherTok{\textless{}{-}}\NormalTok{ student2}
\NormalTok{x}
\end{Highlighting}
\end{Shaded}

\begin{verbatim}
[1] 100  NA  90  90  90  90  97  80
\end{verbatim}

\begin{Shaded}
\begin{Highlighting}[]
\FunctionTok{is.na}\NormalTok{(x)}
\end{Highlighting}
\end{Shaded}

\begin{verbatim}
[1] FALSE  TRUE FALSE FALSE FALSE FALSE FALSE FALSE
\end{verbatim}

\begin{Shaded}
\begin{Highlighting}[]
\NormalTok{x[}\FunctionTok{is.na}\NormalTok{(x)]}
\end{Highlighting}
\end{Shaded}

\begin{verbatim}
[1] NA
\end{verbatim}

\begin{Shaded}
\begin{Highlighting}[]
\FunctionTok{which}\NormalTok{(}\FunctionTok{is.na}\NormalTok{(x))}
\end{Highlighting}
\end{Shaded}

\begin{verbatim}
[1] 2
\end{verbatim}

\begin{Shaded}
\begin{Highlighting}[]
\NormalTok{x}\OtherTok{\textless{}{-}}\NormalTok{student1}
\NormalTok{x}
\end{Highlighting}
\end{Shaded}

\begin{verbatim}
[1] 100 100 100 100 100 100 100  90
\end{verbatim}

\begin{Shaded}
\begin{Highlighting}[]
\NormalTok{x[ }\FunctionTok{is.na}\NormalTok{(x) ] }\OtherTok{=} \DecValTok{0}
\NormalTok{x}
\end{Highlighting}
\end{Shaded}

\begin{verbatim}
[1] 100 100 100 100 100 100 100  90
\end{verbatim}

So far we have a working snippet:

\begin{Shaded}
\begin{Highlighting}[]
\NormalTok{x}\OtherTok{\textless{}{-}}\NormalTok{ student3}
\DocumentationTok{\#\#3 Finds NAs in \textquotesingle{}x\textquotesingle{} and make them 0}
\NormalTok{x[ }\FunctionTok{is.na}\NormalTok{(x) ] }\OtherTok{\textless{}{-}} \DecValTok{0}

\CommentTok{\# finds the min value and rm\textquotesingle{}s it before getting mean}
\FunctionTok{mean}\NormalTok{(x[}\SpecialCharTok{{-}}\FunctionTok{which.min}\NormalTok{(x)])}
\end{Highlighting}
\end{Shaded}

\begin{verbatim}
[1] 12.85714
\end{verbatim}

Now turn it into a function

\begin{Shaded}
\begin{Highlighting}[]
\NormalTok{grade }\OtherTok{\textless{}{-}} \ControlFlowTok{function}\NormalTok{(x) \{}
  \DocumentationTok{\#\#3 Finds NAs in \textquotesingle{}x\textquotesingle{} and make them 0}
\NormalTok{x[ }\FunctionTok{is.na}\NormalTok{(x) ] }\OtherTok{\textless{}{-}} \DecValTok{0}

  \CommentTok{\# drop lowest and find mean}
\FunctionTok{mean}\NormalTok{(x[}\SpecialCharTok{{-}}\FunctionTok{which.min}\NormalTok{(x)])}
\NormalTok{\}}
\end{Highlighting}
\end{Shaded}

\begin{Shaded}
\begin{Highlighting}[]
\FunctionTok{grade}\NormalTok{(student1)}
\end{Highlighting}
\end{Shaded}

\begin{verbatim}
[1] 100
\end{verbatim}

\begin{Shaded}
\begin{Highlighting}[]
\FunctionTok{grade}\NormalTok{(student2)}
\end{Highlighting}
\end{Shaded}

\begin{verbatim}
[1] 91
\end{verbatim}

\begin{Shaded}
\begin{Highlighting}[]
\FunctionTok{grade}\NormalTok{(student3)}
\end{Highlighting}
\end{Shaded}

\begin{verbatim}
[1] 12.85714
\end{verbatim}

Now \texttt{apply()} to our gradebook

\begin{Shaded}
\begin{Highlighting}[]
\NormalTok{gradebook }\OtherTok{\textless{}{-}} \FunctionTok{read.csv}\NormalTok{(}\StringTok{"https://tinyurl.com/gradeinput"}\NormalTok{, }\AttributeTok{row.names =} \DecValTok{1}\NormalTok{)}
\FunctionTok{head}\NormalTok{(gradebook)}
\end{Highlighting}
\end{Shaded}

\begin{verbatim}
          hw1 hw2 hw3 hw4 hw5
student-1 100  73 100  88  79
student-2  85  64  78  89  78
student-3  83  69  77 100  77
student-4  88  NA  73 100  76
student-5  88 100  75  86  79
student-6  89  78 100  89  77
\end{verbatim}

To use \texttt{apply()} function of this \texttt{gradebook}dataset I
need to decide whether I want to apply

\begin{Shaded}
\begin{Highlighting}[]
\FunctionTok{apply}\NormalTok{(gradebook, }\DecValTok{2}\NormalTok{, grade)}
\end{Highlighting}
\end{Shaded}

\begin{verbatim}
     hw1      hw2      hw3      hw4      hw5 
89.36842 76.63158 81.21053 89.63158 83.42105 
\end{verbatim}

\begin{Shaded}
\begin{Highlighting}[]
\NormalTok{ans }\OtherTok{\textless{}{-}} \FunctionTok{apply}\NormalTok{(gradebook, }\DecValTok{1}\NormalTok{, grade)}
\NormalTok{ans}
\end{Highlighting}
\end{Shaded}

\begin{verbatim}
 student-1  student-2  student-3  student-4  student-5  student-6  student-7 
     91.75      82.50      84.25      84.25      88.25      89.00      94.00 
 student-8  student-9 student-10 student-11 student-12 student-13 student-14 
     93.75      87.75      79.00      86.00      91.75      92.25      87.75 
student-15 student-16 student-17 student-18 student-19 student-20 
     78.75      89.50      88.00      94.50      82.75      82.75 
\end{verbatim}

\begin{quote}
Q2 Using your grade() function and the supplied gradebook, Who is the
top scoring student overall in the gradebook? {[}3pts{]}
\end{quote}

\begin{Shaded}
\begin{Highlighting}[]
\FunctionTok{which.max}\NormalTok{(ans)}
\end{Highlighting}
\end{Shaded}

\begin{verbatim}
student-18 
        18 
\end{verbatim}

\begin{Shaded}
\begin{Highlighting}[]
\NormalTok{ans[}\FunctionTok{which.max}\NormalTok{(ans)]}
\end{Highlighting}
\end{Shaded}

\begin{verbatim}
student-18 
      94.5 
\end{verbatim}

\begin{quote}
Q3 From your analysis of the gradebook, which homework was toughest on
students (i.e.~obtained the lowest scores overall? {[}2pts{]}
\end{quote}

\begin{Shaded}
\begin{Highlighting}[]
\NormalTok{masked\_gradebook }\OtherTok{\textless{}{-}}\NormalTok{ gradebook}
\NormalTok{masked\_gradebook[ }\FunctionTok{is.na}\NormalTok{(masked\_gradebook)] }\OtherTok{=} \DecValTok{0}
\FunctionTok{apply}\NormalTok{(masked\_gradebook, }\DecValTok{2}\NormalTok{, mean)}
\end{Highlighting}
\end{Shaded}

\begin{verbatim}
  hw1   hw2   hw3   hw4   hw5 
89.00 72.80 80.80 85.15 79.25 
\end{verbatim}

\begin{Shaded}
\begin{Highlighting}[]
\FunctionTok{which.min}\NormalTok{(}\FunctionTok{apply}\NormalTok{(masked\_gradebook, }\DecValTok{2}\NormalTok{, mean))}
\end{Highlighting}
\end{Shaded}

\begin{verbatim}
hw2 
  2 
\end{verbatim}

I could modify the \texttt{grade()} function to do this too - i would
not drop the lowest options

\begin{Shaded}
\begin{Highlighting}[]
\NormalTok{grade2 }\OtherTok{\textless{}{-}} \ControlFlowTok{function}\NormalTok{(x, }\AttributeTok{drop.low=}\ConstantTok{TRUE}\NormalTok{) \{}

  \DocumentationTok{\#\# Finds NAs in \textquotesingle{}x\textquotesingle{} and make them 0}
\NormalTok{x[ }\FunctionTok{is.na}\NormalTok{(x) ] }\OtherTok{\textless{}{-}} \DecValTok{0}

\ControlFlowTok{if}\NormalTok{(drop.low)\{}
  \FunctionTok{cat}\NormalTok{(}\StringTok{"hello low"}\NormalTok{)}
  \CommentTok{\# drop lowest and find mean}
\NormalTok{out }\OtherTok{\textless{}{-}} \FunctionTok{mean}\NormalTok{(x[}\SpecialCharTok{{-}}\FunctionTok{which.min}\NormalTok{(x)])}
\NormalTok{\} }\ControlFlowTok{else}\NormalTok{ \{}
\NormalTok{  out }\OtherTok{\textless{}{-}} \FunctionTok{mean}\NormalTok{(x)}
  \FunctionTok{cat}\NormalTok{(}\StringTok{"No low"}\NormalTok{)}
\NormalTok{\}}
\FunctionTok{return}\NormalTok{(out)}
\NormalTok{\}}
\end{Highlighting}
\end{Shaded}

\begin{Shaded}
\begin{Highlighting}[]
\FunctionTok{grade2}\NormalTok{(student1,}\ConstantTok{TRUE}\NormalTok{)}
\end{Highlighting}
\end{Shaded}

\begin{verbatim}
hello low
\end{verbatim}

\begin{verbatim}
[1] 100
\end{verbatim}

\begin{quote}
Q4. Optional Extension: From your analysis of the gradebook, which
homework was most predictive of overall score (i.e.~highest correlation
with average grade score)? {[}1pt{]}
\end{quote}

The function to calculate correlations in R is called \texttt{cor()}

\begin{Shaded}
\begin{Highlighting}[]
\FunctionTok{cor}\NormalTok{(ans, masked\_gradebook}\SpecialCharTok{$}\NormalTok{hw1)}
\end{Highlighting}
\end{Shaded}

\begin{verbatim}
[1] 0.4250204
\end{verbatim}

\begin{Shaded}
\begin{Highlighting}[]
\FunctionTok{cor}\NormalTok{(ans, masked\_gradebook}\SpecialCharTok{$}\NormalTok{hw3)}
\end{Highlighting}
\end{Shaded}

\begin{verbatim}
[1] 0.3042561
\end{verbatim}

\begin{Shaded}
\begin{Highlighting}[]
\FunctionTok{cor}\NormalTok{(ans, masked\_gradebook}\SpecialCharTok{$}\NormalTok{hw4)}
\end{Highlighting}
\end{Shaded}

\begin{verbatim}
[1] 0.3810884
\end{verbatim}

\begin{Shaded}
\begin{Highlighting}[]
\FunctionTok{cor}\NormalTok{(ans, masked\_gradebook}\SpecialCharTok{$}\NormalTok{hw5)}
\end{Highlighting}
\end{Shaded}

\begin{verbatim}
[1] 0.6325982
\end{verbatim}

I want to \texttt{apply()} the \texttt{cor()} function over the
\texttt{masked\_gradebook} and use the \texttt{ans} scores for the class

\begin{Shaded}
\begin{Highlighting}[]
\FunctionTok{apply}\NormalTok{(masked\_gradebook, }\DecValTok{2}\NormalTok{, cor, }\AttributeTok{y=}\NormalTok{ans)}
\end{Highlighting}
\end{Shaded}

\begin{verbatim}
      hw1       hw2       hw3       hw4       hw5 
0.4250204 0.1767780 0.3042561 0.3810884 0.6325982 
\end{verbatim}

\begin{Shaded}
\begin{Highlighting}[]
\FunctionTok{which.max}\NormalTok{(}\FunctionTok{apply}\NormalTok{(masked\_gradebook, }\DecValTok{2}\NormalTok{, cor, ans))}
\end{Highlighting}
\end{Shaded}

\begin{verbatim}
hw5 
  5 
\end{verbatim}

\begin{quote}
Q5 Make sure you save your Quarto document and can click the ``Render''
(or Rmarkdown''Knit'') button to generate a PDF foramt report without
errors. Finally, submit your PDF to gradescope. {[}1pt{]}
\end{quote}




\end{document}
